\PassOptionsToPackage{unicode=true}{hyperref} % options for packages loaded elsewhere
\PassOptionsToPackage{hyphens}{url}
%
\documentclass[]{article}
\usepackage{lmodern}
\usepackage{amssymb,amsmath}
\usepackage{ifxetex,ifluatex}
\usepackage{fixltx2e} % provides \textsubscript
\ifnum 0\ifxetex 1\fi\ifluatex 1\fi=0 % if pdftex
  \usepackage[T1]{fontenc}
  \usepackage[utf8]{inputenc}
  \usepackage{textcomp} % provides euro and other symbols
\else % if luatex or xelatex
  \usepackage{unicode-math}
  \defaultfontfeatures{Ligatures=TeX,Scale=MatchLowercase}
\fi
% use upquote if available, for straight quotes in verbatim environments
\IfFileExists{upquote.sty}{\usepackage{upquote}}{}
% use microtype if available
\IfFileExists{microtype.sty}{%
\usepackage[]{microtype}
\UseMicrotypeSet[protrusion]{basicmath} % disable protrusion for tt fonts
}{}
\IfFileExists{parskip.sty}{%
\usepackage{parskip}
}{% else
\setlength{\parindent}{0pt}
\setlength{\parskip}{6pt plus 2pt minus 1pt}
}
\usepackage{hyperref}
\hypersetup{
            pdfborder={0 0 0},
            breaklinks=true}
\urlstyle{same}  % don't use monospace font for urls
\usepackage[margin=1in]{geometry}
\usepackage{graphicx,grffile}
\makeatletter
\def\maxwidth{\ifdim\Gin@nat@width>\linewidth\linewidth\else\Gin@nat@width\fi}
\def\maxheight{\ifdim\Gin@nat@height>\textheight\textheight\else\Gin@nat@height\fi}
\makeatother
% Scale images if necessary, so that they will not overflow the page
% margins by default, and it is still possible to overwrite the defaults
% using explicit options in \includegraphics[width, height, ...]{}
\setkeys{Gin}{width=\maxwidth,height=\maxheight,keepaspectratio}
\setlength{\emergencystretch}{3em}  % prevent overfull lines
\providecommand{\tightlist}{%
  \setlength{\itemsep}{0pt}\setlength{\parskip}{0pt}}
\setcounter{secnumdepth}{0}
% Redefines (sub)paragraphs to behave more like sections
\ifx\paragraph\undefined\else
\let\oldparagraph\paragraph
\renewcommand{\paragraph}[1]{\oldparagraph{#1}\mbox{}}
\fi
\ifx\subparagraph\undefined\else
\let\oldsubparagraph\subparagraph
\renewcommand{\subparagraph}[1]{\oldsubparagraph{#1}\mbox{}}
\fi

% set default figure placement to htbp
\makeatletter
\def\fps@figure{htbp}
\makeatother

\usepackage{setspace}

% this is a lorem ipsum generator for adding dummy texts
\usepackage{lipsum}
\usepackage{tocloft}

% to make the first rows bold in tables
\usepackage{longtable}
\usepackage{tabu}
\usepackage{booktabs}

% this makes list of figures appear in table of contents
\usepackage[nottoc]{tocbibind}

% for passing temporary notes
\usepackage{todonotes}

\usepackage{morefloats}
\usepackage{float}

% highlighting
\usepackage{soul}

% referencing mutliple things with a single command - \cref
\usepackage{cleveref} 

% this makes dots in table of contents
\renewcommand{\cftsecleader}{\cftdotfill{\cftdotsep}}

% to change the title of contents
\renewcommand{\contentsname}{Indhold}

% other translations
\renewcommand{\listtablename}{Tabeloversigt}
\renewcommand\tablename{Tabel}
\renewcommand{\listfigurename}{Figuroversigt}
\renewcommand\figurename{Figur}


% line numbers for review purposes
% this package might not be available in default latex installation 
% get it by 'sudo tlmgr install lineno'
%\usepackage{lineno}
%\linenumbers

% this allows checkmarks in the file
\usepackage{amssymb}
\DeclareUnicodeCharacter{2714}{\checkmark}

% to be able to include latex comments
\newenvironment{dummy}{}{}

% https://gist.github.com/burchill/8873d2ade156b27e92a238a774ce2758
% \usepackage{placeins}

\usepackage{caption}

% Package floatrow Error: Do not use float package with floatrow.
%\usepackage{floatrow}

% Code chunk captions
% 
%\DeclareNewFloatType{chunk}{placement=H, fileext=chk, name=}
%\captionsetup{options=chunk}
%\renewcommand{\thechunk}{Chunk~\thesection.\arabic{chunk}}
%\makeatletter
%\@addtoreset{chunk}{section}
%\makeatother

\author{}
\date{\vspace{-2.5em}}

\begin{document}


\onehalfspacing
\pagenumbering{gobble}

%\begin{titlepage}
\begin{center}
\vspace*{8\baselineskip}
\LARGE{\textbf{Webudvikling med JavaScript Frameworks}}\\
\vspace*{1\baselineskip}
\Large{Emil Martin Vork Heunecke}\\
\vspace*{3\baselineskip}
\Large{\textbf{Vejleder}}\\
Per Bøgeskov\\
\vspace*{2\baselineskip}
Hovedopgave Datamatikeruddannelsen AK\\
University College Lillebælt\\
\end{center}
% \end{titlepage}

\doublespacing

\hypersetup{linkcolor = black}
\newpage
\pagenumbering{roman}
\tableofcontents
\addcontentsline{toc}{section}{\contentsname}

\newpage

% list of figures have to be added manually to table of contents
\listoffigures 

\newpage
\listoftables

\doublespacing

\newpage
\pagenumbering{arabic}
\hypersetup{linkcolor = blue}

\emph{``Learning JavaScript used to mean you weren't a serious software
developer. Today, not learning Javascript means the same thing.''} ---
Tim O'Reilly

\newpage

\hypertarget{basale-teknologier}{%
\section{Basale teknologier}\label{basale-teknologier}}

Dette afsnit handler om det basale teknolgoier

\hypertarget{internettet}{%
\subsection{Internettet}\label{internettet}}

Internettet er et computernetværk bestående af computernetværk som er
forbundet gennem routere. For at sende beskeder på internettet, bruger
computerenhederne TCP/IP protokolstakken. Det er en gruppe af
protokoller udarbejdet af det amerikanske Department of Defense, som
tillader computerenheder at kommunikere. Protokolstakken kan inddeles i
lag for protokollernes forskellige ansvarsområder. TCP/IP modellen
beskriver de fire lag, som protokolstakken består af. Disse lag hedder
\emph{network interface}, \emph{internet}, \emph{transport} og
\emph{application}. Disse kan sammenlignes med lagene i OSI-modellen,
som i sin abstrakte model deler netværkstrafik op i syv lag.

TCP/IP network interface laget svarer til OSI data link laget, som
forbinder internettets netværkstrafik til det lokale netværk. Internet
laget svarer til netværkslaget, som router netværkstrafikken fra
computerenhed til computerenhed. TCP/IP transport laget svarer til OSI
transport laget, som etablerer forbindelsen mellem computerenhederne.
Endelig svarer TCP/IP applikations laget til applikations-,
præsentations- og sessionslagene i OSI-modellen. På dette lag ekisterer
adskillige services som e-mail og \emph{world wide web}.

World wide web blev udviklet i 1989 af Tim Berners Lee. Han
specificerede HTML formatet til at strukturere tekst, Unique Resource
Identifier (URI) specifikationen til at addressere ressourcer og HTTP
protokollen til at kommunikere mellem klienter og servere.

Efterhånden som websider gik fra at være blot at være statiske og
præsentere dokumenter, til at være dynamiske og leve af brugerindhold,
begyndte man at italesætte dette som \emph{web 2.0}. Begrebet blev
opfundet af Tim O'Reilly{[}\^{}web20{]}.

\hypertarget{javascript}{%
\subsection{JavaScript}\label{javascript}}

JavaScript er det programmeringssprog, som flest udviklere
benytter.{[}\^{}so{]} Det er også det programmeringssprog, som har flest
aktive repositories på GitHub.\footnote{\url{https://githut.info/}}
JavaScript anvendes i dag til alt fra simple scripts på websider, til
avancerede webapplikationer, som kan køre i browseren eller på næsten
hvilken som helst anden platform, herunder webservere eller sågar
mikrokontrollere.

\hypertarget{engine-og-runtime-environment}{%
\subsubsection{Engine og Runtime
Environment}\label{engine-og-runtime-environment}}

JavaScript tilhører kategorien af \emph{interpreted} programmeringssprog
- i modsætning til \emph{compiled} programmeringssprog som C\#. På et
compiled programmeringssprog oversætter udviklermaskinen programmets
kildekode til maskinkode med en compiler. I et interpreted
programmeringssprog oversættes kildekoden ikke direkte til maskinkode. I
stedet læser oversætter et andet program kildekoden linje for linje til
maskinkode instruktioner.

JavaScript kildekode oversættes af et JavaScript \emph{engine}. Et
Javacript engine består af en \emph{call stack} og en \emph{memory
heap}. Call stacken er en datastruktur, som holder styr på programmets
\emph{stack frames}, hvilket svarer til de enkeltvise instruktioner i
kildekoden. Fordi JavaScript kun har en enkelt call stack, er JavaScript
\emph{single-threaded}. Det vil sige, at der ikke behandles flere
instruktioner samtidigt. Memory heapen indeholder den allokerede
hukommelse til programmets referenceværdier, hvilket i JavaScript vil
sige objekter.

JavaScript består mere generelt også af det \emph{runtime environment},
som platformen udvider det pågældende engines muligheder med.
Forskellige platforme har forskellige engines og tilbyder forskellige
runtime environments. Det fleste runtime environments har et \emph{event
loop} og en \emph{message queue}, som giver JavaScript mulighed for
\emph{asynkron} single-threaded afvikling af kildekoden.

Asynkrone instrukser i kildekoden er f.eks. dem, som arbejder med
netværkskald og derfor koster tid. For at de ikke skal blokere
applikationens eneste tråd, bliver disse instrukser lagt i en kø
datastruktur. Event loopet holder øje med, om der er flere frames at
køre i call stacken. Hvis ikke, så henter event loopet de asynkrone
instruktioner ud fra message queue og ind i call stack, så de bliver
kørt. På denne måde tillades asynkron JavaScript kode på trods af, at
JavaScript er single-threaded. Browserplatformens runtime environment
har også nogle \emph{web APIer}, som er af særlig interesse for
webudviklere.

Browseren implementerer internt diverse funktioner med et lavere niveau
programmeringssprog som C++. Web APIerne giver adgang til disse
browserfunktioner, til f.eks. netværkskald, lyd eller 3D grafik.
Document APIen er det mest vigtige for webudviklere, da denne giver
adgang til Document Object Model (DOM) træet.

DOM træet er en intern datastruktur i browseren. Datastrukturen er et
hiearkisk træ, som repræsenterer siden aktuelle HTML struktur. Hver
eneste knude i DOM træet repræsenterer et enkelt HTML element med alle
dets attributter. Hvert undertræ består altså af et HTML element og alle
dets børn. DOM træet er til hver tid en afspejling af HTML strukturen på
den aktuelle webside. Med Document APIen er det muligt at manipulere DOM
træet og alle dets knuder, f.eks. ved at oprette nye HTML elementer
eller ved at oprette event listeners på eksisterende HTML elementer.

\hypertarget{versioner}{%
\subsubsection{Versioner}\label{versioner}}

Det oprindelige JavaScript blev skabt i 1995 af Brendan Eich, som var
ansat hos Netscape, for at understøtte scripting i deres browser,
Netscape Navigator 2.0. Det blev udgivet med en pressemeddelse\footnote{\url{https://web.archive.org/web/20070916144913/http://wp.netscape.com/newsref/pr/newsrelease67.html}}
hvori intentionen med sproget blev lagt frem. JavaScript blev
præsenteret som et nemanvendeligt \emph{object scripting} sprog til
webdokumenter, som supplement til HTML og Java. Sproget blev også
præsenteret som en implementation af en \emph{åben standard} under en
kommercielt venlig licens, for at tillade det brede udviklersamfund at
bidrage til sprogets udvikling.

JavaScript betegnes som sagt et objekt scripting sprog, og man kalder
også JavaScript \emph{objekt baseret}. JavaScript understøtter det
objekt-orienterede programmeringsparadigme, men sproget adskiller sig
fra konventionelt objekt-orienterede sprog som Java og C\# bl.a. ved
ikke at have statiske typer eller sondring mellem instanser og klasser.

Den fornævnte åbne standard udmøntede sig i JavaScript Reference
Implementation (JSRef), som var et open source JavaScript engine, der
udviklede sig til . Men eftersom JSRef ikke blev udgivet i tide, havde
Microsoft i mellemtiden analyseret JavaScript som det var implementeret
i Netscape Navigator, og på den baggrund udviklet deres egen engine, som
de kaldte Chakra. Fordi navnet JavaScript var
varemærkeregistreret\footnote{\url{http://tsdr.uspto.gov/\#caseNumber=75026640\&caseSearchType=US_APPLICATION\&caseType=DEFAULT\&searchType=statusSearch}},
blev Microsofts udgave af det samme sprog kaldt JScript.

For at harmonisere de forskellige JavaScript engines, blev JavaScript
indsendt til ECMA International, som håndterer webstandarder. De udgav i
1997 en åben specifikation af sproget i form af ECMAScript
standarden\footnote{\url{https://www.ecma-international.org/ecma-262/}}.
Dermed kunne konkurrende JavaScript engines overholde en fælles
standard, så alle JavaScript engines kunne mindske jævnbyrdes afvigelser
i håndteringen af JavaScript kildekode. Der findes i dag mange
forskellige JavaScript engines til forskellige platforme, som alle
afvikler JavaScript kode efter denne samme standard. ECMAScript
standarden hedder formelt ECMA-262, og implementeringerne af standarden
i form af programmeringssprog og engines hedder formelt ECMAScript i
stedet for JavaScript pga. varemærkebeskyttelsen.

I daglig tale og i denne hovedopgave er ECMAScript synonymt med
JavaScript, selvom det faktisk stadig er et registreret varemærke, der i
dag er eget af Oracle, som endda i sjældne tilfælde\footnote{\url{https://adtmag.com/articles/2018/04/18/javascript-trademark.aspx}}
stadig beskytter deres ophavsret på navnet. Der var på et tidspunkt tale
om, at Oracle skulle donere varemærket til ECMA International\footnote{\url{https://twitter.com/BrendanEich/status/986605049987002368}}
i erkendelse af, at navnet JavaScript de facto var blevet et offentligt
domæne, men dette blev aldrig til noget.

\begin{quote}
``ECMAScript was always an unwanted trade name that sounds like a skin
disease'' --- Brendan Eich
\end{quote}

ECMAScript standarden dikterer altså udviklingen af JavaScript sproget.
Det betyder, at ældre browsere ikke understøtter alle moderne JavaScript
funktioner, fordi de bruger et JavaScript engine, som implementerer en
ældre udgave af ECMA-262. Der har været flere store udgivelser af
ECMA-262, hvoraf de første tre udkom hurtigt efter hinanden: ES1 i 1997,
ES2 i 1998 og ES3 i 1999. ES4 var under udvikling længe, men blev af
forskellige årsager droppet. Først ti år efter udgivelsen af ES3, kom
efterfølgeren ES5 i 2009, hvilket blev en milepæl for JavaScript, og det
er den seneste udgave af sproget, som er understøttet af alle mainstream
browsere inkl. Internet Explorer. Seks år senere kom ES6 i 2015, og
herefter begyndte ECMA at udgive en ny version hver juni, så disse
udgaver navngives typisk efter deres udgivelsesår: den syvende udgave
hedder ES2016, den ottende ES2017, osv.

\hypertarget{vuxe6rdier-og-variable}{%
\subsubsection{Værdier og variable}\label{vuxe6rdier-og-variable}}

JavaScript har overordnet to forskellige datatyper af værdier. Det er
\emph{primitiver}, som opbevares på stack, og \emph{objekter}, som
opbevares på heap.

Primitiver er simple værdier, som JavaScript har syv forskellige typer
af: \texttt{Number}, \texttt{Boolean}, \texttt{String}, \texttt{Null},
\texttt{Undefined}, \texttt{Symbol} (introduceret i ES6) og
\texttt{BigInt} (introduceret i ES7).

Objekter består af \emph{properties}, som er navngivne primitiver eller
objekter. Properties kan også indeholde \emph{functions}, som i
Javascript er specielle objekter af typen \texttt{Function}, og
\emph{arrays} af primitiver og/eller objekter, som i JavaScript også er
specielle objekter, der har typen \texttt{Array}. Functions og arrays er
altså som næsten alt andet i JavaScript objeker.

JavaScript har som forventeligt variable til at indeholde værdier. I
JavaScript kan ikke bare værdien af variabel ændre sig, når programmet
kører, men også variablens type. Man kan lave variable med nøgleordene
\emph{var}, \emph{let} eller \emph{const}. Det er muligt at lave flere
variable kun med brug af et enkelt nøgleord. Variabler oprettet med
\emph{var} kan ligesom

JavaScript kører som sagt i et runtime environment, som gør nogle
specifikke objekter tilgængelige for udvikleren. Alle browsere giver fx
adgang til \texttt{document} objektet, som er udviklerens indgang til
DOM træet. Den giver metoder til at manipulere det hierarki af objekter,
som hver især repræsenterer de HTML tags, som udgør den aktuelle DOM.

\hypertarget{objekt-orienteret-programmering}{%
\subsubsection{Objekt-orienteret
programmering}\label{objekt-orienteret-programmering}}

JavaScript er et objektorienteret sprog, men det har (i modsætning til
fx. C\#) ikke klasser og instanser, bare objekter. For at kunne lave
objekter af en bestemt type, bruger man, i stedet for klasser og
instanser, en \emph{constructor funktion}. Denne specielle metode
navngives konventionelt med stort begyndelsesbogstav. På constructor
funktionen er der en property \texttt{prototype}, som indeholder et
objekt. Ved at sætte properties på prototype objektet, angiver man
hvilke felter, som skal være på objekter skabt med constructor
funktionen. Når man laver objekter ved at kalde constructor funktionen
med nøgleordet \texttt{new}, vil man kunne tilgå de properties og
metoder, man har angivet i prototypen.

Når constructor funktionen kaldes med \texttt{new}, får Prototype
objektet som standard en \texttt{constructor} property, som peger
tilbage på den constructor funktion, som prototypen tilhører. For
eksempel peger \texttt{Animal.prototype.constructor} tilbage på
\texttt{Animal}. og \texttt{\_\_proto\_\_} property, som angiver hhv.
typens constructor funktion, og constructor funktionen for den nedarvede
type (i sidste ende \texttt{Object}).

Hvis man vil specialisere sin constructor function, fx. lave en
\texttt{Mammal} som specialisering af \texttt{Animal}, laver man en ny
constructor function, og sætter prototypen til en instans af overtypen.

Når man tilgår en property, fx. (???) vil JavaScripts engine søge efter
en property med det navn: først blandt objektets properties, dernæst i
dens prototypes properties, så prototypens \textbf{proto}'s properties,
osv. indtil roden af nedarvingshiearkiet, som er constructor funktionen
\texttt{Object}.

I ES6 blev en ny syntaks indført for at gøre det lettere at arbejde med
prototypiske nedarvningskæder. Man kan nu bruge \texttt{class}
nøgleordet til at definere sine \emph{klasser}. Syntaksen for klasser i
JavaScript minder meget om klasser i andre objektorienterede sprog, fx.
C\#, men resultaten af klasse definitionen minder mest om prototypisk
nedarvning, så man kan næsten betragte sprogkonstruktionen som
syntaktisk sukker for at definere en constructor med dens prototype
felter og metoder. Det forklarer bl.a. hvorfor, at klasser i JavaScript
endnu ikke officielt understøtter private feltvariable.

\hypertarget{funktionel-programmering}{%
\subsubsection{Funktionel
programmering}\label{funktionel-programmering}}

Jeg havde da skrevet en masse fedt, hvor blev det af\ldots{}

\hypertarget{moderne-syntaks}{%
\subsubsection{Moderne syntaks}\label{moderne-syntaks}}

JavaScript er navngivet efter Java, fordi det blev skabt med ideen om at
være et supplement til Java. Derfor er syntaksen også inspireret af
Java, og nogle vil sige mere generelt af C. Hovedformålet var dog, at
det skulle være nemt for nybegyndere at tage sproget og syntaksen til
sig. Efterhånden som JavaScript blev mere udbredt, opstod behovet for at
udvikle sproget og syntaksen. Især ES6 introducerede nogle vigtige nye
sprogfunktioner og syntaks.

\hypertarget{lambda-udtryk}{%
\paragraph{Lambda udtryk}\label{lambda-udtryk}}

\hypertarget{moduler}{%
\paragraph{Moduler}\label{moduler}}

I JavaScript er det muligt at importere \emph{moduler}. Et modul er en
kodefil, som kan importeres af andre kodefiler. Det nuværende
modulsystem i JavaScript blev udgivet med ES6.

Moduler tillader at eksportere \emph{named exports}, så man f.eks. kun
eksporterer én funktion fra en fil. Der er også mulighed for **default
exports*, som kan importeres nemmere, og generelt bruges, hvis moduler
har en primær funktionalitet, som skal fremhæves. Default exports
behøver ikke navngives, da de navngives ved importeringen. Endelig Før i
tiden var der ingen understøttelse af moduler, så problemet blev løst
med kodebiblioteker som CommonJS og RequireJS (hvorfor bruges requirejs
stadig i Node?)

ES6 moduler

\hypertarget{udbredte-teknologier}{%
\section{Udbredte teknologier}\label{udbredte-teknologier}}

\hypertarget{vuxe6rktuxf8jer-og-udviklingsmiljuxf8er}{%
\subsection{Værktøjer og
udviklingsmiljøer}\label{vuxe6rktuxf8jer-og-udviklingsmiljuxf8er}}

For meget nutidig JavaScript udvikling gælder det, at man ikke bare er
nødt til at kende til web platformen og dens teknologier, men der
eksisterer også mange værktøjer til at løse flere problemer, end
JavaScript umiddelbart er designet til.

\hypertarget{node}{%
\subsubsection{Node}\label{node}}

Node er et runtime environment til at køre JavaScript på serveren i
stedet for klienten. Node bruger samme JavaScript engine som Chrome
browseren, men giver adgang til andre APIer i stedet for browserens web
APier. Med Nodes APIer har man fx adgang til computerens filsystem. Man
kan afvikle JavaScript kildekode på alle styresystemer med Node
installeret, hvilket gør det velegnet til cross-platform udvikling af
f.eks. webservere.

Node bruger modulsystemet CommonJS, fordi Node udkom før ES modulerne
blev introduceret i ES6. De indbyggede APIer i Node er moduler, som man
indlæser med CommonJS. For eksempel kan man få adgang til filsystemet
med modulet \texttt{fs}, som indlæses med den indbyggede
\texttt{require} funktion.

Node installeres sammen med \emph{NPM}, som er et program til at styre
et projekts afhængigheder. NPM samarbejder med et online register af
JavaScript biblioteker. Ved brug af NPM kan man nemt installere
biblioteker fra registeret til sit projekt og styre afhængihederne med
semantisk versionsstyring. Node installeres også sammen med NPX, som kan
køre biblioteker, der er installeret som afhængighed, eller hente
biblioteket fra registeret for at køre det, og herefter fjerne det igen.

\hypertarget{webpack}{%
\subsubsection{Webpack}\label{webpack}}

Webpack er det, man kalder en \emph{module bundler}. Det vil sige, at
Webpacks primære formål er på buildtime at samle koden til én samlet
JavaScript \emph{bundle}, som indeholder alle applikationens moduler og
deres indbyrdes afhængigheder. I konfigurationsfilen for Webpack
definerer man et \emph{entry point}, som Webpack skal starte i. Webpack
starter i den fil, man har defineret som entry point, og gennemløber
derfra kildekodens import/export udtryk, som den undervejs bruger til at
bygge en \emph{dependency graph}, som repræsenterer alle
webapplikationens ES6 moduler og deres indbyrdes afhængigheder.

Med udgangspunkt i denne dependency graph, samler Webpack alle ES6
modulerne til en samlet JavaScript kildekode, som er klar til at blive
downloadet og præsenteret af browseren. Dette giver i sig selv en
gevinst, for brugeren kan få hele webapplikationen i ét request. Det kan
gøre indlæsningen af alle modulerne betydeligt hurtigere på en
traditionel HTTP/1.1 forbindelse med høj forsinkelse, da HTTP/1.1
håndterer hvert request synkront og dermed blokerer renderingen indtil
hver static asset er downloaded og eksekveret.

Webpack kører også forskellige optimeringer på kildekoden, når bundler,
for eksempel for at mindske filstørrelsen af den endelige bundle. Alt
dette gør webpack ud af boksen med sin standard konfiguration, men langt
de fleste bruger deres egen webpack konfiguration skræddersyet til deres
projekt. Med webpacks konfigurationsfil er det muligt at ændre alle

For at komme i gang med at bruge webpack, er Node en forudsætning. Man
kan bruge NPM til lave et nyt tomt projekt og installere Webpack:

Hvis overstående konfigurationsfil er til stede i rodmappen, vil Webpack
automatisk samle den op, når man kører kommandoen \texttt{webpack} fra
projektets rodmappe som arbejdsmappe. Hvis der findes en
\texttt{main.js}, vil Webpack køre og resultere i en bundle i
\texttt{./dist/bundle.js}.

For at ændre output filnavnet til fx. \texttt{output.js}, tilføjer man
en \texttt{output} property til \texttt{config} objektet, med en
\texttt{filename} property. For at placere output placeringen til fx.
\texttt{./build/}, tilføjer man yderligere en \texttt{path} property til
\texttt{output} objektet.

Webpack giver mulighed for, at man kan indlæse trejdeparts
\emph{loaders} til at håndtere andre filtyper end JavaScript. Man
konfigurerer webpack til at håndtere forskellige filtyper med
forskellige loaders. Hver loader er en preprocessor, som Webpack
anvender, når den støder på en angiven filtype. Loaders kan bl.a. sørge
for, at Webpack også kan bundle stylesheets og andre assets. En hyppig
anvendelse af loaders er også at få Webpack til at samarbejde med andre
værktøjer som Babel og TypeScript.

\hypertarget{babel}{%
\subsubsection{Babel}\label{babel}}

Babel Overstående HTML-lignende syntaks er den syntaksudvidelse, som
kaldes JSX. Det transpileres (af Webpack) med en loader (Babel) til
standard JavaScript. Webpack er en modul bundler, som samler projektets
ES 6 moduler (ved import/export statements) til én samlet kildekode, og
med Babel oversætter speciel syntaks som f.eks. JSX til standard
JavaScript. Overstående kode bliver oversat af Babel til noget lignende:

\hypertarget{typescript}{%
\subsubsection{TypeScript}\label{typescript}}

TypeScript er en udvidelse af JavaScript, som udvider JavaScript med
type annotationer. TypeScript introducerer syntaks, man kender fra
typestærke sprog som f.eks. interfaces, enums og generics. TypeScript
introducerer også syntaks, man ikke traditionelt kender fra typestærke
objekt-orienterede sprog, men til gengæld fra typestærke funktionelle
sprog, som f.eks. \emph{mapped types} og \emph{algebraiske data typer}.

Mapped types er en syntaks, der kan transformere en type definition til
en anderledes tilstand. F.eks. mapper
\texttt{Partial\textless{}T\textgreater{}} typen alle properties på den
generiske type til at være optional.

Algebraiske data typer er en syntaks, som tillader at kombinere typer
for at skabe nye typer.

Fordelene ved TypeScript er mange, men vigtigst er naturligvis den
typesikkerhed, som man kan opnå med TypeScript på compile-time.
Typesikkerhed er uvurderligt, når man skal refaktorere gammel kode, da
det giver mulighed for at ændre koden med en vis grad af tillid til, at
der hvert fald ikke er introduceret nye fejl pga. inkompatible typer.

En anden fordel TypeScript har er, at det typestærkheden forbedrer
dokumentation for koden. Hvis man importerer JavaScript moduler som har
en korrekt type definition, kan man med sin IDE f.eks. udforske modulets
eksporterede objekter, metoder, interfaces, etc. uden at man behøver
nærlæse eller afvikle kildekoden. Med DefinitelyTyped registret, er det
muligt at finde korrekte type definitioner til de fleste JavaScript
kodebiblioteker, som endnu ikke har en officiel type definition (.d.ts
fil).

En af fordelene ved TypeScript er også, at det ligesom Babel er en
transpiler, der oversætter syntaks, og man ser ofte nye syntaks
funktioner i de nyeste versioner af TypeScript, før de kommer til
ECMAScript.

TypeScripts type system kaldes \emph{structural typing}, hvilket
adskiller sig fra f.eks. C\#'s nominal typing ved at være mindre streng,
når TypeScript skal bestemme hvorvidt et objekt overholder en type
definition. Det bestemmes udelukkende ved, om et objekt har properties,
som er tilsvarende alle de properties, som type definitionen indeholder,
om den har yderligere properties, en anden type annotation men de samme
properties, eller måske slet ingen type annotation. Det er kun de
properties, som udgør måltypens interface, der bliver taget i
betragtning.

\hypertarget{biblioteker-og-frameworks}{%
\subsection{Biblioteker og frameworks}\label{biblioteker-og-frameworks}}

Som udviklere er det vores opgave at vælge de rigtige biblioteker og
frameworks til ethvert givet projekt. Dette kan blive kompliceret, fordi
det ikke kun er projektets umiddelbare behov, der betyder noget. Man er
del af et team og et større socialt system i en virksomhed, som har
behov for at kunne vedligeholde systemet uden store omkostninger ved
omskoling af sin medarbejderstab.

Mulighederne for at vælge JavaScript værktøjer er mange. Det er enormt
mange JavaScript biblioteker tilgængelige. Man kan dele de tilgængelige
biblioteker op i grupper af \emph{generelle} biblioteker,
\emph{specialiserede} biblioteker og \emph{frontend framework}
biblioteker.

Der er generelle biblioteker, som giver mange hjælpemetoder til at gøre
diverse opgavere lettere eller mindre repetetive. Et eksempel på det
kunne være JQuery (\textasciitilde{}90kb), som i lang tid var et
populært bibliotek til at gøre udvikleren i stand til at udtrykke mere
med færre linjer JavaScript.

Der er specialiserede biblioteker, som indeholder relativt få linjer
kode og løser et meget specifikt problem. Et eksempel kunne være Redux
(\textasciitilde{}2kb), som kun har ansvar for at håndtere
applikationstilstand.

Endelig er der biblioteker, der fungerer som frontend frameworks. Selvom
det er muligt at opbygge en simpel frontend kun med HTML, CSS og
JavaScript, giver forskellige frontend frameworks mulighed for bl.a. at
strukturere kildekoden, så den ikke bliver for kompleks og
uvedligeholdbar. Der er også behov for en struktureret måde at håndtere
komponenter, styling og applikationstilstand. Frontend frameworks
opstiller på forskellig vis et stillads til at løse disse problemer på
en mere eller mindre fleksibel måde.

Frontend frameworks vil for denne opgave betragtes som værende
biblioteker, der varierer i størrelse, og evt. bruges sammen med andre
biblioteker, med det formål at bygge og vedligeholde \emph{single page
applications}. Et eksempel kunne være Angular (\textasciitilde{}500kb) ,
som er en ``alt-i-en'' løsning til at bygge single page applications,
der kan håndtere applikationstilstand med den indbyggede data-binding i
biblioteket. Et andet eksempel kunne være Vue(\textasciitilde{}80kb),
som ofte bruges sammen med Vuex(\textasciitilde{}10kb) biblioteket til
at håndtere applikationstilstand.

Single page applications er en tilgang til webapplikationer, som samler
hele hjemmesidens indhold på en webside, i stedet for at sprede det ud
på flere hyperlinkede websider. Når brugeren af webapplikationen
navigerer rundt på siden, bliver det efterspurgte indhold hentet
dynamisk ned fra webserveren og vist på websiden uden genindlæsning.
Tilgangen er i kraftig vækst{[}\^{}spa-growth{]} og bliver brugt af
store virksomheder som Facebook, Twitter og Google. En meget udbredt
arkitektur for single page applications, er en JavaScript SPA, der
kommunikerer vha. HTTP og \emph{JSON} med en \emph{RESTful} webservice.

JSON er en data formatterings standard til at beskrive strukturen af
JavaScript objekter. Selvom JSON er navngivet efter JavaScript, er det
et selvstændigt data format, som kan bruges af alle programmeringssprog.
Syntaksen for JSON tilsvarer JavaScript object og array literal
syntaksen. Man kan repræsentere både primitiver, objekter og arrays med
JSON.

RESTful webservices er karakteriseret ved, at de giver webadgang til
ressourcer, som fx kan ligge i en database. Hver resource kan
identificeres fra dens URL-adresse. Man manipulerer ressourcerne med
HTTP beskedens operationsverbum: GET, PUT, POST eller DELETE som bruges
til henholdvis at læse, opdatere, oprette eller slette ressourcer.

\hypertarget{popularitet}{%
\subsubsection{Popularitet}\label{popularitet}}

Eftersom det, for de fleste webudviklere, ikke er overkommeligt at lære
alle tilgængelige populære frontend frameworks, bør man gøre nogle
overvejelser, før man vælger hvilket frontend framework at lære. For
denne opgave vil jeg se på de mest populære frontend frameworks blandt
udviklere og arbejdsgivere.

Når man dykker ned i de seneste års udvikling indenfor SPA frameworks,
opstår der et mønster. Blandt udviklere er de mest populære frontend
frameworks React, Vue og Angular.\footnote{\url{https://www.jetbrains.com/lp/devecosystem-2019/javascript/}}
Det er umuligt at forudse, hvilke af disse frontend frameworks, der
overlever længst, eller hvornår de bliver erstattet af noget nyt. Men
der tegner sig et billede af, at flere udviklere benytter og udtrykker
tilfredshed med Vue og React end Angular. React ligger i 2019 placeret
på andenpladsen af mest anvendte biblioteker (med jQuery fortsat på
førstepladsen). I kategorien ``mest elskede'' frontend frameworks, er
React den mest populære, med Vue lige i hælene, mens Angular halter
bagefter på en niendeplads. Vue har flere stjerner på dets GitHub
repository end React og Angular.

Hvis man ser på, hvor meget de respektive frontend frameworks downloades
af udviklere, ligger React væsentligt højere end de andre med op imod
seks millioner downloads om ugen, sammenlignet med to millioner for
Angular og en million for Vue.

\hypertarget{react}{%
\subsubsection{React}\label{react}}

React er et JavaScript kodebibliotek til at bygge brugergrænseflader
baseret på komposition af komponenter. Alting i React er en komponent,
og selv hele applikationen er en komponent, som er sammensat af de
øvrige komponenter. Komponent strukturen gør det muligt at genbruge og
indkapsle alt lige fra en simpel checkbox til en hel applikation. Det
gør det også nemt at ændre komponenterne og f.eks. tilføje yderligere
state, uden at det skaber en dominoeffekt af side effekter i resten af
applikationen. Komponent modellen passer ikke så godt med objekt
modellen, og principper vi kender som søjler i objekt orienteret
programmering, er ikke anvendelige eller har bedre løsninger i React.
Facebook promoverer stærkt ''Composition over Inheritance'', og udtaler
at de aldrig har brugt nedarvning i deres tusindevis af komponenter.
{[}komponenter-billede{]}

React koncentrerer sig kun om brugergrænsefladen, og det er både en
fordel og en ulempe. Langt de fleste React projekter vil have behov for
at hente yderligere tredjeparts pakker ned for at håndtere problemer,
som f.eks. Angular har indbyggede løsninger på. React omtales ofte
heller ikke som et framework men et kodebibliotek, og man sammenligner
det ofte med View-laget i MVC. Det gode ved Reacts minimalistiske
tilgang er, at der er stor fleksibilitet i hvordan man løser et givent
problem, og der er mange højt specialiserede værktøjer til at håndtere
nicher som f.eks. routing eller global state.

React har hvad man kalder en deklarativ API. I stedet for at beskrive
hvert trin og i hvilken rækkefølge, man opdaterer de enkelte elementer
på brugergrænsefladen (imperativt), så beskrives hvordan man ønsker
(deklarerer), at brugegrænsefladen skal se ud afhængigt af dens
tilstand.

\hypertarget{komponenter-og-props}{%
\paragraph{Komponenter og props}\label{komponenter-og-props}}

En React komponent er ikke andet end en funktion, som returnerer noget,
der i bund og grund bare er HTML. Strukturen og indholdet kan evt.
afhænge af nogle funktionsparametre -- eller hvad man i React termer
ville kalde props. Disse props følger et princip om unidirectional data
flow, eller en-vejs databinding. Props kan kun sendes fra
forældre-komponent til barn-komponent.

Det hænger sammen med en anden egenskab ved props, som er, at de skal
være immutable. En komponent kan læse sine props, men ikke overskrive
dem. Eftersom en komponent modtager sine props som input, betyder det at
man undgår side effekter i sine komponenter. Med andre ord, vil de samme
props som input altid generere det samme HTML som output. Man siger, at
komponenten er en pure function med hensyn til sine props. Reacts eneste
regel er, at alle komponenter skal være pure functions mht. sine props.
For at en funktion kan være en pure function, må den for det a) ikke
have sideeffekter b) ikke kalde non-pure functions (som Date.now) og c)
behandle sine argumenter som immutable (readonly). Der findes JavaScript
kodebiblioteker som ImmutableJS til at håndhæve dette, eller code
linting værktøjer som ESLint.

Props kan indeholde callbacks, så en forældre-komponent kan sende en
funktion ned til barn-komponenten, som barn-komponenten kan kalde med en
eller flere parametre for at sende data til forældre-komponenten. Det er
den eneste umiddelbare måde komponenter kan sende data på, til andre end
sine børn. Selvom det er med til at gøre programmet nemt at debugge og
forstå, virker det umiddelbart som en begrænsning. For eksempel når når
to komponenter A og B langt fra hinanden i træhiearkiet skal bruge data
fra hinandens props. Det kræver at en fælles forældre-komponent til
komponent A og komponent B sender dataen ned i sine props gennem flere
mellemled. Dette pattern kaldes \emph{lifting-state-up}.

Når man har en stor applikation med mange komponenter, som deler mange
props, kan det blive uoverskueligt at vedligeholde med at løfte
tilstanden, da det overholder ikke DRY princippet, eftersom den manuelle
proces med at sende dataen igennem komponenten via props skal gentages
for hvert enkelt mellemled mellem A og B. Dette anti-pattern kaldes
\emph{prop-drilling}.

\hypertarget{redux}{%
\paragraph{Redux}\label{redux}}

En måde at omgå denne ``prop-drilling'' er ved at benytte Reacts
indbyggede Context API. Den tillader komponenter at dele props indenfor
en kontekst uden at mappe props mellem komponenter.

En anden mere avanceret måde, er at gøre som Facebook og bruge software
design mønstret Flux. Flux applikationer har tre hovedbestanddele:
dispatcher, stores og views. Ideen med Flux er, at den delte state
ligger i en single source of truth dvs. i stores. State i stores kan kun
opdateres ved at trigge en action. Actions fortæller sine stateændringer
(payload) til dispatcher. Stores lytter på ændringer fra dispatcher, og
opdaterer deres egen state i overeensstemmelse hermed.

Redux har også mulighed for middleware, som kan manipulere actions på
vej ind og ud af reduceren. Dette giver mulighed for f. eks. asynkrone
actions til API kald med redux-thunk middlewaren. Redux API'en giver
også mulighed for at kombinere reducers med en higher order reducer
funktion kaldet combineReducers. Det er altså en funktion, der tager
flere reducers som parameter, og returnerer en ny kombineret reducer.
Typisk har man en reducer slice for hver feature / domæne i projektet,
som kombineres til en enkelt root reducer med combineReducers. Root
reduceren bruges af Redux til at lave store objektet.

For at forbinde sine React komponenter til sin Redux store anbefales det
at bruge Redux API'ens connect funktion. Connect tager to funktioner som
input og returnerer en funktion, som er en higher order component, der
tager en React komponent som input og returnerer en ''forbedret'' udgave
af komponenten med adgang til store og action creators via props.

\hypertarget{client-side-og-server-side-rendering}{%
\paragraph{Client-side og server-side
rendering}\label{client-side-og-server-side-rendering}}

Alt React kode som har direkte med browserens DOM at gøre, er splittet
op i sit eget ReactDOM bibliotek. Man kan også udvikle med React mod en
anden målplatform end browseren, ved at erstatte ReactDOM med et
kodebibliotek til f.eks. cross-platform mobilapps (React Native) og
konfiguration.

ReactDOM fungerer ved, at der fra webapplikationens index side kører en
JavaScript fil i browseren, som holder browserens DOM synkroniseret med
applikationens komposition af React komponenter og state. Alt logikken
til at generere HTML dynamisk er således flyttet over i
klienten/browseren. Denne client-side-rendering gør det muligt at have
en interaktiv hjemmeside uden brug af en webserver.

Selvom man har en SPA uden en webserver, kan man stadig have
pseudorouting med JavaScripts History API eller React udgaven, React
Router. Men det er altid index siden, der indlæser hele
webapplikationen, som præsenterer indholdet ud fra URL routen. Det
medfører en bedre brugeroplevelse, da klienten aldrig indlæser en ny
side efter index siden, og dermed slipper for nogensinde at genindlæse
siden.

React opdaterer kun de DOM elementer i browserens hukommelse, som det er
nødvendigt for at matche det tilsvarende React elements state. Denne
optimisering fremkommer af Reacts VDOM. Det er en virtuel kopi af
DOM'en, som eksisterer med den nyeste opdaterede state i hukommelsen.
Før React opdaterer DOM, hvilket er en meget dyr operation, sammenligner
meget hurtige algoritmer VDOM med DOM og finder de steder, hvor der er
forskel i browserens DOM state fra Reacts interne VDOM state, så kun de
absolut nødvendige DOM elementer behøver at blive opdateret

React understøtter også server-side-rendering, hvor serveren foretager
den første indlæsning af webapplikationen og sender et snapshot af det
fuldt formede HTML dokument til klienten. På denne måde understøtter man
bedre søgemaskine crawlerne og opnår dermed en bedre rank i
søgemaskinen. Man mindsker også tiden før first meaningful paint (når
man kan se sidens primære indhold).

\hypertarget{innovative-teknologier}{%
\section{Innovative teknologier}\label{innovative-teknologier}}

\hypertarget{innovation}{%
\subsection{Innovation}\label{innovation}}

Everett Rogers har beskrevet udbredelsen af innovative teknologier i et
socialt samfund. Hans \emph{technology adoption lifecycle} model
beskriver fem faser i anvendelse af ny teknologi, med udgangspunkt i de,
som tager teknologien til sig.

Den første gruppe til at tage en ny teknologi i brug kaldes
\emph{innovators} (innovatør). Når teknologien ikke længere er ukendt,
kaldes den næste gruppe \emph{early adopters}. Når en teknologi begynder
at blive meget udbredt, kaldes de som tager teknologien i brug
\emph{early majority}. Når en teknologi er nær toppen af sin
popularitet, kaldes denne gruppe \emph{late majority}. Den sidste gruppe
er \emph{laggards}, der anvender en teknologi, som i de flestes øjne er
forældet.

Det er mest risikofyldt at være innovatør, fordi der på dette stadie
kræver en dybdegående forståelse at vurdere, om den nye teknologi har
potentiale. Der er også størst potentiel gevinst ved at være innovatør,
da man kan bruge den innovative teknologi til at udkonkurrere sine
konkurrenter. De fleste ledere af store virksomheder vurderer, at deres
virksomhed ville blive udkonkurreret inden for syv år, hvis de ikke
formår at følge med innovationen.\footnote{\url{https://devops.com/survey-sees-massive-adoption-of-microservices/}}.

Ron Adner har beskrevet en iboende konflikt mellem innovatøren og
\emph{forbrugeren}. Her forstås innovatøren som producenten af en ny
innovation, som skal sælge sin innovation til forbrugeren som kunde. Man
kan også forestille sig udvikleren som innovatør, der vil overbevise
ledelsen i en virksomhed som forbruger. Man kan også forestille sig den
nyansatte juniorudvikler på et team som innovatøren, der vil overbevise
seniorudviklerne som forbrugeren. Konflikten består af en forskel i
opfattelse af \emph{cost} (omkostninger) og \emph{benefit} (fordele) ved
at tage en innovation til sig.

Innovatøren tænker på omkostninger med hensyn til den umiddelbare
omkostning, der er ved at tage teknologien i brug. Forbrugeren tænker på
omkostninger i forhold til den umiddelbare omkostning plus alle de andre
ændringer, som er nødvendige for at bruge innovationen.

Innovatøren tænker på fordelene med hensyn til den absolutte fordel, som
forbrugeren modtager. Forbrugeren tænker over benefit i forhold til den
relative fordel, som innovationen giver sammenlignet med de tilgængelige
alternativer.

Innovatøren har altså en tendens til at sammenligne den absolutte fordel
med den umiddelbare omkostning, mens forbrugeren sammenligner den
relative fordel med den samlede omkostning.

Denne forskel kan optræde i scenariet med udvikleren i virksomheden, der
ikke ser virksomhedens samlede omkostninger (omskoling af
medarbejderstab, indkøb af licenser, etc.) ved at tage en innovation til
sig. På samme måde kan den nyansatte juniorudvikler have nemmere ved at
tage en innovativt teknologi i brug, fordi hans omkostningsperspektiv er
mindre end seniorudvikleren, der har samlet meget viden om tidligere
teknologier.

\hypertarget{micro-frontends}{%
\subsection{Micro Frontends}\label{micro-frontends}}

Software arkitektur har længe bevæget sig mod at splitte monolitiske
systemer op i mindre moduler. I stedet for at alle udviklere i en
virksomhed arbejder på samme, komplekse system, har mange virksomheder
set gevinster i microservice arkitekturen. Her består det samlede system
af en lang række undersystemer, som kan udvikles og udgives uafhængigt
af hinanden.

Microservice arkitekturen er blevet meget udbredt, og størstedelen af
store virksomheder er enten ved at tage microservices til sig, eller
bruger allerede udelukkende microservices\footnote{\url{https://www.leanix.net/en/blog/beyond-agile-time-to-adopt-microservices}}.
Brugergrænsefladen bliver for mange alligevel stadig udviklet som et
monolitisk system. Microfrontend arkitekturen er at overføre ideen om
microservices til brugergrænsefladen.

Microfrontends kommer med mange af de samme fordele som microservices.
Den centrale fordel er muligheden for en fastlægge en individuel kode-,
test- og driftproces for hver microfrontend.

Dermed kan mindre teams være ansvarlige for egne microfrontends, og
hurtigere komme med inkrementelle opdateringer uden at vente på
hinanden. Det understøtter en agil procesmodel, hvor hvert team autonomt
kan komme med en ny version af deres microfrontend efter et sprint.

man kan opdele microfrontends i ``vertikale'' hensyn til
forretningsdomænet, i stedet for ``horisontale'' hensyn til tekniske
ansvarsområder. Dermed får hvert team en bedre helhedsforståelse af det
domæne, som de er ansvarlige for med deres microfrontend.

Med microfrontends er kildekoden for hver microfrontend afkoblet fra
hinanden. Fordi undersystemet er mindre end hele systemet, vil
kildekoden for hver microfrontend også være mindre og nemmere at
vedligeholde. Hvert team har også bedre fleksibilitet i deres valg af
biblioteker og frameworks, som passer til lige præcis deres
microfrontend.

\hypertarget{module-federation}{%
\subsection{Module Federation}\label{module-federation}}

I beta version 5 af Webpack er der en ny funktion, som spiller en vigtig
rolle i at understøtte en microfrontend arkitekturen. Det er indtil
videre kun i beta version, og bliver officielt udgivet sammen med
Webpack 5. Module federation tillader at bundle hver eneste individuelle
microfrontend med alle sine afhængigheder. Dermed kan man have en
individuel proces/pipeline for at udvikle, teste og udgive hver enkelt
microfrontend uafhængigt af hinanden, men alligevel indlæse dem dynamisk
med JavaScript i runtime og bruge dem som en samlet helhed.

Men module federation giver også mulighed for, at man angiver
\emph{shared dependencies}. Hvis man f.eks. har angivet \texttt{react}
som \texttt{shared} for to forskellige microfrontends, og indlæser begge
microfrontends, vil kun den ene af de to tilsvarende afhængigheder blive
indlæst i runtime, men til gengæld vil dén ene afhængighed blive delt af
begge microfrontends. Dermed opnår man \emph{decoupling} af de enkelte
microfrontends på en ressourceffektiv måde.

Module federation bruger betegnelsen \emph{hosts} for applikationer, der
indlæser en \emph{remote} applikation. En applikation kan fungere som
både host og remote på samme tid.

Man angiver hosts og remotes i sin \texttt{webpack.config.js}
konfigurationsfil. Module federation er inkluderet i Webpack som et
plugin, der først skal indlæses med RequireJS:

ModuleFederationPlugin initialisers i \texttt{config} objektets
\texttt{plugins} array med et konfigurationsobjekt som parameter. Her
angiver man den nuværende applikations navn, hvis den skal indlæses som
remote. Her angiver man også hvilke remotes, som Webpack skal forvente
at indlæse i denne host, og hvilke interne moduler, som skal udstilles
som remote.

Hermed vil Webpack håndtere det korrekt, i stedet for at melde fejl,
hvis den støder på et \texttt{import} udtryk, som skal findes på en
anden webadresse i stedet for det lokale filsystem.

Hvis man ønsker at eksponere dele af applikationen som remote, er det
for øjeblikket nødvendigt at angive applikationens offentlige
webadresse, som den udgives på, i \texttt{output} property på
\texttt{config} objektet til Webpack. Der arbejdes på en løsning, hvor
dette kan angives fra konsummenten (hosten) eller udledes automatisk fra
importeringen af remote.

Når man herefter kører Webpack, vil den som buildprodukt også lave en
lille fil ved navn \texttt{remoteEntry.js}. Denne fil er
host-applikationens indgangspunkt til remote-applikationen. Den
indeholder ikke selve de udstillede moduler, men eksisterer for at
indlæse. Denne fil skal refereres til i host'en, fx ved at indsætte et
\texttt{\textless{}script\textgreater{}} tag i
\texttt{\textless{}head\textgreater{}} tagget af websidens
\texttt{index.html}:

Hermed er det muligt at importere alle de moduler, som
remote-applikationen udstiller, så længe applikationen er i drift på den
angivne webadresse. Man bruger en dynamisk import, hvor det navn, som
man angiver for sin remote-applikation i host-applikationens
\texttt{ModuleFederationPlugin}, fungerer som et namespace, som om
remote-applikationen ligger på det lokale filsystem.

\end{document}
